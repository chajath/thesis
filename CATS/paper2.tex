%% LyX 2.0.4 created this file.  For more info, see http://www.lyx.org/.
%% Do not edit unless you really know what you are doing.
\documentclass[british,fleqn]{CRPITStyle}
\usepackage[T1]{fontenc}
\usepackage[latin9]{inputenc}
\usepackage{listings}
\lstset{comment={[l]{\#}},
commentstyle={\color{purple}\ttfamily},
frame=leftline,
identifierstyle={\color{black}},
keywords={new, true, false, function, return, if, while, else, output, input},
keywordstyle={\bfseries\underline},
ndkeywords={this, global},
ndkeywordstyle={\bfseries},
numbers=left,
numbersep=10pt,
numberstyle={\tiny},
sensitive=false,
tabsize=2,
xleftmargin=20pt}
\usepackage{geometry}
\geometry{verbose,tmargin=1cm,bmargin=1cm,lmargin=1cm,rmargin=1cm,headheight=1cm,headsep=1cm,footskip=0cm}
\setcounter{secnumdepth}{3}
\setcounter{tocdepth}{3}
\usepackage{color}
\usepackage{amsthm}
\usepackage{amsmath}
\usepackage{amssymb}

\makeatletter
%%%%%%%%%%%%%%%%%%%%%%%%%%%%%% Textclass specific LaTeX commands.
\numberwithin{equation}{section}
\numberwithin{figure}{section}
\theoremstyle{plain}
\newtheorem{thm}{\protect\theoremname}
  \theoremstyle{definition}
  \newtheorem{defn}[thm]{\protect\definitionname}
  \theoremstyle{definition}
  \newtheorem{example}[thm]{\protect\examplename}
  \theoremstyle{plain}
  \newtheorem{lem}[thm]{\protect\lemmaname}
  \theoremstyle{plain}
  \newtheorem{cor}[thm]{\protect\corollaryname}

%%%%%%%%%%%%%%%%%%%%%%%%%%%%%% User specified LaTeX commands.
\usepackage{stmaryrd}
\usepackage{syntax}
\usepackage{color}
\definecolor{lightgray}{rgb}{.9,.9,.9}
\definecolor{darkgray}{rgb}{.4,.4,.4}
\definecolor{purple}{rgb}{0.65, 0.12, 0.82}

\AtBeginDocument{\catcode`\_=8}

\newcommand{\dd}[1]{\llbracket#1\rrbracket}
\newcommand{\bind}{\Yright}

\exhyphenpenalty=10000\hyphenpenalty=10000

\newcommand\conferencenameandplace{19th Computing: Australasian Theory Symposium (CATS 2013), Adelaide, Australia, January 2013}
\newcommand\volumenumber{xxx}
\newcommand\conferenceyear{2013}
\newcommand\editorname{Anthony Wirth}

\makeatother

\usepackage{babel}
  \providecommand{\corollaryname}{Corollary}
  \providecommand{\definitionname}{Definition}
  \providecommand{\examplename}{Example}
  \providecommand{\lemmaname}{Lemma}
\providecommand{\theoremname}{Theorem}

\begin{document}

\title{Monadic Theory of Language Analysis and its Application in Type Analysis
of Dynamic Language}


\author{In-Ho Yi, Peter Schachte and Harald S�ndergaard\\
Department of Computing and Information Systems \\
 The University of Melbourne, Victoria 3010, Australia \\
 i.yi@student.unimelb.edu.au, schachte@unimelb.edu.au, harald@unimelb.edu.au}

\maketitle
\submittedstandard
\begin{abstract}
We present a novel approach in construction and formalisation of a
static analysis of a language, in which we formulate a polymorphic
and parametric interpretation of a language common to both denotational
semantics and static analysis. We apply this in analysing a duck-typing,
reflective and function currying language. We show the benefit of
this new approach in term of its terseness and modularity, and a new
research methodology in which we gradually build language features
and analysis on top of a previous incarnation of the theory.
\end{abstract}

\section{Introduction}

There are two distinct needs for the development of the present work.
First, there is a theoretician's need to have a simple and elegant
way of presenting a formalised argument of a static analysis which
enables one to focus on what is at the heart of the analysis - definitions
of abstract domain and their relations to the concrete domain of a
language semantics. Denotational semantics is a step towards such
direction. In this paper we endeavour to further modularise the system
to the point where formal presentation of semantics and analysis can
share a common framework that mostly deals with linguistic aspects
of a programming language.

Second, there is a practical need of analysing dynamic languages that
are finding more popularity in recent years. Traditionally these languages
were termed ``scripting'' languages, as they were mainly used for
automating tasks and processing strings. However, with the advent
of Web environment, languages such as Perl and PHP gained popularity
as languages for web application development. On a client side, web
pages make heavy use of JavaScript, a dynamic typing language, to
deliver dynamic contents to the browser. Recent years have seen an
increasing use of JavaScript in a server side environment as well.

What these languages provide is an ability to rapidly prototype and
validate application models in a real time REPL environment. Another
strength comes from the fact that programmers do not need to have
a class structure defined upfront. Rather, class structures and types
of variables in general are dynamically built. This reduces an initial
overhead of software design.

However, these features come at a cost. Lacking formal and static
definition of type information makes dynamically typing languages
harder to analyse. This difficulty causes several practical problems. 
\begin{itemize}
\item As applications become more mature, more efforts are devoted to program
unit testing or assertions to make sure type safety of systems. These
extra overheads can sometimes overshoot the benefit of having a dynamically
typing language.
\item Whereas programmers writing statically typed languages enjoy the abundance
of development tools, choices of tools for aiding development of dynamically
typed languages are limited, and lacking much of the power, largely
due to the difficulty and sometimes infeasibility of conduct a type
analysis.
\item Not having a static type structure has a significant performance implication.
\end{itemize}
With these problems in mind, we design a model language that has a
dynamism comparable to those in the aforementioned scripting languages,
such as duck-typing, reflection and function currying. A notable omission
is a closure scoping. However, allowing function currying gives enough
expressive power to the language.

The two concerns are not distinct ones, but an interconnected dialectic.
Theoretical need is there because of a difficulty of describing abstract
and concrete meaning of dynamic languages which often allow side-effect
causing, type-altering functions.


\section{Related works}

On a theoretical side, denotational semantics is the starting point
of our development of a theoretical system. The idea of incorporating
monads in the denotational semantics, as presented in \cite{Liang:phd1998,Liang_Hudak:ESOP1996},
directly inspired the present work. We extend the idea to the static
analysis of a language.

An idea of constructing formulae with parametric types can be found
in \cite{Wadler:1989:TF:99370.99404}. This work is a special application
of the parametricity in the field of language semantics and analysis.

On the type analysis of dynamic languages, \cite{Anderson:ECOOP2005,Anderson:ENTCS2005,Guha:ESOP2011,Wrigstad:POPL2010}
devise model languages and their semantics for the purpose of static
analysis of dynamic languages. A major difference between them and
the model language presented in this paper is that our model language
is designed to capture the feature of real world languages that functions
alter types through a side-effect causing statements. Similarities
and differences of this work compared to the cited works are given
as we encounter them in this paper.

Type analysis plays a crucial part in compiling scripting languages,
mainly to improve performance. \cite{Ancona:2007:RST:1297081.1297091,Dufour:2006}
design restricted versions of a scripting language so that a static
inference of types can be performed. Some techniques employed in those
projects can also be found in this paper, the use of allocation sites
as static references is an example.

An important use case of functions in dynamic typing language is ``mixin''
(\cite{Bracha:1990:MI:97946.97982}) functions, where, by passing
argument to such function the variable, the object is extended with
extra methods. There are model languages and formalisations of mixin
functions (\cite{Anderson:ECOOP2005,Vub-prog-tr96aformalisation}).
Where those works seek to find a functional models for mixins, we
model a language with side-effect causing functions that is expressive
enough to program a mixin inheritance.

\cite{Jensen:SAS2009} built a feature-complete analyser for JavaScript
language. Our work can be progressed further to become a theoretical
counterpart of the analyser. Furthermore, such attempt at formalising
the analysis might pave a way for a further refinement and improvement.


\section{Analytic framework}

We first describe a monadic theory of language analysis. In this section
we produce a particular monadic construct, specifically designed for
the purpose of analysis. We then examine key properties of such construct
that is vital to our theory.


\subsection{State collection}

We design a system that collects state transitions from one to another,
as has occurred in a particular execution point. We augment a syntax
tree by giving each nodes its unique ID. Then, the type of the end
product that the system is collecting can be described as thus:
\begin{defn}[State Collection]


$StateCol:Stmt\rightarrow State\rightarrow State$
\end{defn}

\subsection{Monad}

We define the monadic constructions. We define a type constructor
$M$ and a bind operator $\bind$.
\begin{defn}[Type constructor]
 Type constructor $M$ gives the following polymorphic type.

\begin{multline*}
M\ a=Func\rightarrow\left(State\times StateCol\right)\rightarrow\\
\left(State\times StateCol\times a\right)
\end{multline*}

\begin{defn}[Bind operator]
We define a bind operator $\bind$ as following:

$\left(\bind\right):M\ a\rightarrow\left(a\rightarrow M\ b\right)\rightarrow M\ b$

$T\ \bind\ U=\lambda f,r,s.U\ t\ f\ r'\ s'$ where $\left\langle r',s',t\right\rangle =T\ f\ r\ s$
\end{defn}
\end{defn}
Now we define a series of lift functions in order to provide modularity.
For example, if a particular parametrised operation takes a state
but only produces a value, it would be redundant to include a state
as a part of returning type, to match the definition of monadic binding.
In such case, we take a function that returns only a value, then lift
such function to be used in the monadic binding.
\begin{defn}[Lift functions]
We define following lift functions to incorporate non-monadic functions
as a part of monadic transformation:
\begin{itemize}
\item $I_{A}$ is an identity state transformer that takes a constant and
lifts it to an identity state transformer with the constant as a return
value
\end{itemize}

\begin{gather*}
I_{a}:a\rightarrow b\rightarrow M\ a\\
I_{a}\ v=\lambda f,r,s.\left\langle r,s,v\right\rangle 
\end{gather*}

\begin{itemize}
\item $I_{T}$ lifts a state transformer with a return value to a monadic
function
\end{itemize}

\begin{gather*}
I_{S}:\left(State\rightarrow State\times a\right)\rightarrow M\ a\\
I_{S}\ t=\lambda f,r,s.\left\langle r',s,a\right\rangle \ \mbox{where}\ \left\langle r',a\right\rangle =t\ r
\end{gather*}

\begin{itemize}
\item $I_{V}$ lifts a function that takes a state and returns a value to
a monadic function
\end{itemize}

\begin{gather*}
I_{V}:\left(State\rightarrow a\right)\rightarrow M\ a\\
I_{V}\ t=\lambda f,r,s.\left\langle r,s,t\ r\right\rangle 
\end{gather*}

\begin{itemize}
\item $I_{0}$ takes a state transformation with no return value and attaches
a meaningless return value to it
\end{itemize}

\begin{gather*}
I_{0}:\left(State\rightarrow State\right)\rightarrow M\ a\\
I_{0}\ t=\lambda f,r,s.\left\langle r',s,0\right\rangle \ \mbox{where}\ \left\langle r'\right\rangle =t\ r
\end{gather*}


\end{defn}
Now we turn to examine properties of the monad. \cite{Liang:phd1998}
gives a survey of mathematical properties that monadic constructs
exhibit. For the purpose our discussion, we limit ourselves to the
discussion of properties of monads that are crucial in proving correctness
of the system.
\begin{thm}[Continuity of bind operation]
\label{thm:bindmono}Given two monotonous functions $F$ and $G$,
$F\bind G$ is also a monotonous function.\end{thm}
\begin{proof}
For state and state collection arguments, $F\bind G$ is a series
of monotonic transformation. Since the composition of two monotonic
functions is also monotonic, state and state collection are monotonic.

For the result of $F$ passed on to $G$ as a bound variable, a return
value from $F$ is monotonous to the input. The return value, then
becomes part of input to the $G$ function, whose return value is
also monotonous.\end{proof}
\begin{thm}
Given two pairs of isomorphic functions $\left\langle F,F'\right\rangle $
and $\left\langle G,G'\right\rangle $, $F\bind G$ and $F'\bind G'$
are also isomorphic.\end{thm}
\begin{proof}
Let $F\left(\rho_{0}\right)=\left\langle \rho_{1},\alpha_{1}\right\rangle $,
$G\left(\rho_{1},\alpha_{1}\right)=\left\langle \rho_{2},\alpha_{2}\right\rangle $,
$F'\left(\eta_{0}\right)=\left\langle \eta_{1},\upsilon_{1}\right\rangle $,
$G'\left(\eta_{1},\upsilon_{1}\right)=\left\langle \eta_{2},\upsilon_{2}\right\rangle $

If there are defined relationship $\prec$between two different domains
that two pairs of functions are isomorphic with, isomorphicity of
$F$ and $F'$ provides that $\rho_{0}\prec\eta_{0}$, $\rho_{1}\prec\eta_{1}$
and $\alpha_{1}\prec\upsilon_{1}$. Isomorphicity of $G$ and $G'$
gives $\rho_{2}\prec\eta_{2}$ and $\alpha_{2}\prec\upsilon_{2}$.
Therefore $F\bind G$ and $F'\bind G'$ are also isomorphic.
\end{proof}
Finally, we define a state collection helper function $T_{S}$, which
takes a state transformation and appends to the transformation a new
association of state transformation from initial state to the final
state for a given statement.
\begin{defn}[State collection helper function]


\begin{alignat*}{2}
 & T_{S} & : & M\ a\rightarrow s\rightarrow M\ a\\
 & T_{S}\ X\ z & = & \lambda f,r,s.\left\langle r',s'',a\right\rangle \\
 &  &  & \begin{aligned}\mbox{where} & \left\langle r',s',a\right\rangle =X\ f\ r\ s\\
 & s''=s'\left[z\times r\mapsto r'\right]
\end{aligned}
\end{alignat*}


\end{defn}
We now have monadic constructs and auxiliary functions to describe
a semantic functions of the model language.


\section{Model language}

In this section we define a model language, ``Simple Duck-Typing
Language (SDTL)''. Our model language features C-like syntax and
resembles JavaScript. The key differences are:
\begin{itemize}
\item Absence of prototypes
\item Lack of lexical scoping
\item Function currying
\item Explicit use of ``$\mathtt{global}$'' keyword
\end{itemize}
It is our anticipation that, with the methodology that we employ,
we can gradually extend this language to embrace much of the language
features in real world.

Following is the syntax of SDTL.

\begin{grammar}

<con> ::= Number | Boolean

<Lexpr> ::= ID | <Expr> `.' ID

<Expr> ::= <con> | `global' | `this'
\alt  <Lexpr>
\alt  `new' <Lexpr> `(' [<Expr> [,<Expr>]*]? `)'
\alt  <Lexpr> `(' [<Expr> [,<Expr>]*]? `)'
\alt  <Expr> <binop> <Expr>
\alt  `(' <Expr> `)'

<binop> ::= `+' | `-' | `*' | `/' | `>' | `<' | `=='

<Stmt> ::= nil | <Stmt> `;' <Stmt> | <Expr>
\alt `output' <Expr>
\alt <LExpr> `=' <Expr>
\alt `if' `(' <Expr> `)' `{' <Stmt> `}'
\alt `if' `(' <Expr> `)' `{' <Stmt> `}' `else' `{' <Stmt> `}'
\alt `while' `(' <Expr> `)' `{' <Stmt> `}'
\alt `function' ID `(' [ID [, ID]*]? `)' `{' <Stmt> `}'
\alt `return' <Expr>

\end{grammar}

$\mathtt{global}$ is a reference to a global object, and the reference
to global object is program invariant (akin to `window' in Client-side
javascript or `global' in Node.js). When the number of arguments given
is less than that of the arity of the function as declared, a function
is partially applied, and a curried function is returned for a further
currying or invoking. In SDTL, like many other dynamic languages,
functions are first-class objects. As such, we can express mixins
in this language:

\begin{lstlisting}[name=juice]
function Fruit(v) {
	this.value = v;
}

function juicible(fruit, juice) {
	function juiceMe(j,x) {
		return this.value + j + x;
	}
	fruit.juice = juiceMe(juice); #currying
	global.answer=42;
}

global.answer = 0;

# Juicibles
apple = new Fruit(15);
juicible(apple, 20);
grape = new Fruit(30);
juicible(grape, 50);

# Non-juicibles
banana = new Fruit(20);
watermelon = new Fruit(25);

output apple.juice(10); # 15 + 20 + 10
output grape.juice(10); # 30 + 50 + 10
output global.answer; # 42
\end{lstlisting}


`juicible' function turns an ordinary fruit object into a juicible
object. Although constructed by the same constructor function, types
of variables `apple' and `grape' is different to that of non-juicibles.
This gets further complicated if we add in-line extensions:

\begin{lstlisting}[name=juice]
function cider(x,y) {
	return this.juice(x) + y;
}

apple.cider = cider;
output apple.cider(20,50); # 15+20+20+50
\end{lstlisting}


Now object type `apple' is different to that of `grape'. In our analysis,
we approximate the object types of a variable by treating each construction
site as a separate object type. Similar approach has been taken in
\cite{Jensen:SAS2009}.


\section{Analysis}

We now present a concrete and abstract analysis of SDTL. We first
discuss briefly on analysing functions in SDTL. Then we present a
parametrised monadic semantics and concrete and abstract functions
that we can plug-in to obtain denotational semantics and a static
analysis. Finally, we use properties of monadic constructs to prove
the correctness of the analysis.


\subsection{Functions}

A difficulty of analysing a function in a reflective language without
any given type information has been noted in \cite{Bono:arXiv2012}.
Functions in SDTL cause side-effects, which in turn allow them to
function as a class extender and a mixin creator. $\mbox{JS}_{0}$
language, as developed and presented in \cite{Anderson:ECOOP2005,Anderson:ENTCS2005}
is an attempt to contain a mixin creating behaviour in a functional
model language. Similar approach has been taken in the discussion
of $\lambda_{S}$ language as devised in \cite{Guha:2011:TLC:1987211.1987225}.

With the languages like JavaScript, it is in the nature of such language
that functions are side-effect causing, and much of the type operations
are being done with by the side-effects. If we model a function in
such a manner, then the function itself is a state transformer of
a type: $State\rightarrow State\times a$. Parameters and a receiver
object can be a part of the state as we define. This formulation could
have been further refined by distinguishing what remains the same
throughout a function execution (a receiver object, for example),
and those that can be transformed. However, we do not make such distinction
for simplicity's sake.

Observe that in a given program, we have a mutually recursive collection
of such functions. In order to analyse such collection, we augment
the syntax tree of the SDTL language to give each function declaration
a unique ID. We then map to each unique ID the function as we have
discussed. Given recursions, we formulate a fixed point of such collection
with mutual reference.

\[
\mathcal{F}_{D}=Y\lambda f\lambda n.\mathcal{S}_{D}\ \mathtt{Stmt}_{n}\ f
\]


Where $\mathcal{S}_{D}$ is a denotational semantic function that
takes a statement and a function fixed point, and returns an updated
state and a return value. With this formulation in mind, we define
$\mathcal{F}$ and other semantic functions.


\subsection{Semantic functions}

As discussed earlier, definition of semantics functions has a state
collector incorporate to cater for the needs of the analysis.
\begin{defn}[Semantic functions]
We define a monadic denotational semantics of the SDTL program

\begin{eqnarray*}
\mathcal{F} & : & f\rightarrow\left(State\times StateCol\right)\rightarrow\\
 &  & \left(State\times StateCol\times a\cup\left\{ Void\right\} \right)\\
\mathcal{S} & : & Stmt\rightarrow s\rightarrow\mathcal{F}\rightarrow\left(State\times StateCol\right)\rightarrow\\
 &  & \left(State\times StateCol\times a\cup\left\{ Void\right\} \right)\\
\mathcal{E} & : & Expr\rightarrow e\rightarrow\mathcal{F}\rightarrow\left(State\times StateCol\right)\rightarrow\\
 &  & \left(State\times StateCol\times a\right)\\
\mathcal{L} & : & LExpr\rightarrow\mathcal{F}\rightarrow\left(State\times StateCol\right)\rightarrow\\
 &  & \left(State\times StateCol\times a\right)
\end{eqnarray*}


Definition for those functions are given in figure \ref{semmon}.
\end{defn}
\begin{figure*}
\begin{align*}
 & \mathcal{F} & = & Y\lambda f.\lambda n.\mathcal{S}\ \mathtt{Stmt}_{n}\ \mathtt{sid}\left(\mathtt{Stmt}_{n}\right)\ f\\
 & \mathcal{S}\dd{S_{1};S_{2}} & = & T_{S}\ \left(\begin{aligned}S\ S_{1}\ \mathtt{sid}\left(S_{1}\right) & \bind\lambda r.\\
\mathtt{isterm}\left(r,S\ S_{2}\ \mathtt{sid}\left(S_{2}\right)\right)
\end{aligned}
\right)\\
 & \mathcal{S}\dd{\mathtt{return}\ e} & = & T_{S}\ \left(\mathcal{E}\ e\ \mathtt{eid}\left(e\right)\right)\\
 & \mathcal{S}\dd{\mathtt{if}\left(e\right)S_{1}} & = & T_{S}\ \left(\begin{aligned}\mathcal{E}\ e\ \mathtt{eid}\left(e\right) & \bind\lambda v.\\
\mathtt{cond}\left(v,\mathcal{S}\ S_{1}\ \mathtt{sid}\left(S_{1}\right),I_{\alpha}\ Void\right)
\end{aligned}
\right)\\
 & \mathcal{S}\dd{\mathtt{if}\left(e\right)S_{1}\mathtt{else}S_{2}} & = & T_{S}\ \left(\begin{aligned}\mathcal{E}\ e\ \mathtt{eid}\left(e\right) & \bind\lambda v.\\
\mathtt{cond}\left(v,\mathcal{S}\ S_{1}\ \mathtt{sid}\left(S_{1}\right),\mathcal{S}\ S_{2}\ \mathtt{sid}\left(S_{2}\right)\right)
\end{aligned}
\right)\\
 & \mathcal{S}\dd{id=e} & = & T_{S}\ \left(\begin{aligned}\mathcal{E}\ e\ \mathtt{eid}\left(e\right) & \bind\lambda v.\\
\mathtt{asg}\left(id,v\right) & \bind\lambda\_.\\
I_{\alpha}\ Void
\end{aligned}
\right)\\
 & \mathcal{S}\dd{e_{0}.id=e_{1}} & = & T_{S}\ \left(\begin{aligned}\mathcal{E}\ e_{0}\ \mathtt{eid}\left(e_{0}\right) & \bind\lambda v.\\
\mathcal{E}\ e_{1}\ \mathtt{eid}\left(e_{1}\right) & \bind\lambda r.\\
I_{T}\ \mathtt{set}\left(r,id,v\right)
\end{aligned}
\right)\\
 & \mathcal{S}\dd{\mathtt{while}\left(e\right)S_{1}}\ f\ \rho_{0} & = & T_{S}\ \left(Y\lambda x.\lambda\rho,r.\mathtt{isterm}\left(r,\left(\begin{aligned}\mathcal{E}\ e\ \mathtt{eid}\left(e\right)\bind\lambda v.\\
{\scriptstyle \begin{gathered}\mathtt{cond}\left(v,\left(\begin{gathered}\lambda f,\rho'.\\
x\ \left(\mathcal{S}\ S_{1}\ \mathtt{sid}\left(S_{1}\right)f\ \rho'\right)
\end{gathered}
\right),\left(I_{\rho}\ Void\right)\right)\end{gathered}
}
\end{aligned}
\right)\ f\right)\ \rho_{0}\ {\scriptstyle Void}\right)\\
 & \mathcal{S}\dd{\mathtt{output}\ e} & = & T_{S}\ \left(\begin{aligned}\mathcal{E}\ e\ eid\left(e\right) & \bind\lambda v.\\
\mathtt{output}\left(v\right)
\end{aligned}
\right)\\
 & \mathcal{E}\dd{con}\ \mathtt{eid} & = & I_{\alpha}\ \mathtt{conval}\left(con\right)\\
 & \mathcal{E}\dd{\mathtt{global}}\ \mathtt{eid} & = & I_{V}\ \mathtt{obj}\left(0\right)\\
 & \mathcal{E}\dd{L}\ \mathtt{eid} & = & \mathcal{L}\ L\\
 & \mathcal{E}\dd{\mathtt{this}}\ \mathtt{eid} & = & I_{V}\ \mathtt{getthis}\\
 & \mathcal{E}\dd{\mathtt{input}}\ \mathtt{eid} & = & I_{T}\ \mathtt{getinput}\\
 & \mathcal{E}\dd{\mathtt{new}\ L\left(E*\right)}\ \mathtt{eid} & = & \left(\begin{aligned}\mathcal{L}\ L & \bind\lambda n.\\
\mathtt{evalParams}\ n\ E*\ \phi & \bind\lambda p.\\
I_{T}\ \mathtt{newobj} & \bind\lambda m.\\
\mathtt{call}\left(n,p,m,\mathtt{eid}\right) & \bind\lambda\_.\\
I_{\alpha}\ m
\end{aligned}
\right)\\
 & \mathcal{E}\dd{L\left(E*\right)}\ \mathtt{eid} & = & \left(\begin{aligned}\mathcal{L}\ L & \bind\lambda n.\\
\mathtt{evalParams}\ n\ E*\ \phi & \bind\lambda p.\\
\mathtt{call}\left(n,p,\mathtt{obj}\left(0,\rho\right),\mathtt{eid}\right)
\end{aligned}
\right)\\
 & \mathcal{E}\dd{e_{0}.L\left(E*\right)}\ \mathtt{eid} & = & \left(\begin{aligned}\mathcal{E}\ e_{0}\ \mathtt{eid}\left(e_{0}\right) & \bind\lambda t.\\
\mathcal{L}\ L & \bind\lambda n.\\
\mathtt{evalParams}\ n\ E*\ \phi & \bind\lambda p.\\
\mathtt{call}\left(n,p,t,\mathtt{eid}\right)
\end{aligned}
\right)\\
 & \mathcal{E}\dd{e_{0}\ binop\ e_{1}}\ \mathtt{eid} & = & \left(\begin{aligned}\mathcal{E}\ e_{0}\ \mathtt{eid}\left(e_{0}\right) & \bind\lambda c_{0}.\\
\mathcal{E}\ e_{1}\ \mathtt{eid}\left(e_{1}\right) & \bind\lambda c_{1}.\\
I_{V}\ \mathtt{bin}\left(binop,c_{0},c_{1}\right)
\end{aligned}
\right)\\
 & \mathcal{L}\dd{id} & = & \left(I_{V}\ \mathtt{val}\left(id\right)\right)\\
 & \mathcal{L}\dd{\mathtt{this}} & = & \left(I_{V}\ \mathtt{getthis}\left(\rho\right)\right)\\
 & \mathcal{L}\dd{\mathtt{global}} & = & \left(I_{V}\ \mathtt{obj}\left(0\right)\right)\\
 & \mathcal{L}\dd{e_{0}.id} & = & \left(\begin{aligned}\mathcal{E}\ e_{0}\ \mathtt{eid}\left(e_{0}\right) & \bind\lambda v.\\
I_{V}\ \mathtt{get}\left(v,id\right)
\end{aligned}
\right)
\end{align*}


Where

$\mathtt{Stmt}_{n}$ is a function body of a function numbered $n$

$\mathtt{evalParam}$ is a parameter evaluation auxiliary function,
defined as following:

\begin{gather*}
\mathtt{evalParam}:\left[ID\right]\times\left[Expr\right]\times\left[a\right]\times\mathcal{F}\times State\times StateCol\rightarrow State\times StateCol\times\left[a\right]\\
\mathtt{evalParam}\ \phi\ ps\ f\ \rho\ \sigma=\left\langle \rho,\sigma,ps\right\rangle \\
\mathtt{evalParam}\ \left[e|Exprs\right]\ ps\ f\ \rho\ \sigma=\mathtt{evalParam}\ Exprs\ \left[ps|con\right]\ f\ \rho_{1}\ \sigma_{1}\\
\ \mbox{where}\left\langle \rho_{1},\sigma_{1},con\right\rangle =\mathcal{E}\ e\ f\ \rho\ \sigma
\end{gather*}


\caption{Semantic functions}
\label{semmon}

\end{figure*}



\subsection{Denotational semantics}

We define a concrete domain and concrete functions to plug into the
semantic functions, the end result of which is a state-collecting
denotational semantics.
\begin{defn}[Concrete domain]
We define a concrete domain of a concrete state $CState$ as following:

\begin{eqnarray*}
Value & = & \mathbb{Z}\cup\left\{ T,F\right\} \cup Objects\cup\\
 &  & FunctionPointers\\
Objects & = & \left\{ OBJ_{n}\mid n\in\left\{ 0\right\} \cup\mathbb{N}\right\} \\
FunctionPointers & = & \left\langle FP_{n},\left[Value\right]\right\rangle \\
CState & = & \left\langle Vars,ObjMem,This\right\rangle \\
Vars & = & id\rightarrow Value\\
ObjMem & = & n\rightarrow V\\
This & = & Objects
\end{eqnarray*}

\begin{defn}[Concrete functions]
We define concrete functions in figure \ref{concfunc}.
\end{defn}
\end{defn}
\begin{figure*}


\begin{eqnarray*}
\mathtt{isterm}\left(v,s\right) & = & \lambda f,\rho.\begin{cases}
s\ f\ \rho & v=Void\\
I_{\rho}\ v & \mbox{otherwise}
\end{cases}\\
\mathtt{cond}\left(v,s_{1},s_{2}\right) & = & \lambda f,\rho.\begin{cases}
s_{1}\ f\ \rho & v=true\\
s_{2}\ f\ \rho & v=false
\end{cases}\\
\mathtt{asg}\left(id,v\right) & = & \lambda\left\langle V,OM,T\right\rangle .\left\langle V\left[id=v\right],OM,T\right\rangle \\
\mathtt{val}\left(id\right) & = & \lambda\left\langle V,OM,T\right\rangle .V\left(id\right)\\
\mathtt{get}\left(OBJ_{n},id\right) & = & \lambda\left\langle V,OM,T\right\rangle .OM\left(n\right)\\
\mathtt{set}\left(OBJ_{n},id,v\right) & = & \lambda\left\langle V,OM,T\right\rangle .\left\langle V,OM\left[OBJ_{n}\left[id\mapsto v\right]\right],T\right\rangle \\
\mathtt{conval}\left(con\right) & = & con\\
\mathtt{obj}\left(n\right) & = & \lambda\left\langle V,OM,T\right\rangle .OM\left(n\right)\\
\mathtt{getthis} & = & \lambda\left\langle V,OM,T\right\rangle .T\\
\mathtt{getinput\left(\rho\right)} & = & \mbox{see note}\\
\mathtt{dooutput}\left(v\right) & = & \mbox{see note}\\
\mathtt{newobj} & = & \lambda\left\langle V,OM,T\right\rangle .\left\langle \left\langle V,OM\left[n\mapsto\phi\right],T\right\rangle ,OBJ_{n}\right\rangle \ \mbox{where}\ n=N\left(OM\right)\\
\mathtt{call}\left(\left\langle FP_{n},C\right\rangle ,p,t,\_\right) & = & \lambda f,s,\left\langle V,OM,T\right\rangle .\begin{cases}
\begin{gathered}\left\langle \left\langle V,OM',T\right\rangle ,s',r\right\rangle \ \mbox{where}\ \left\langle \left\langle V',OM',T'\right\rangle ,s',r\right\rangle =\\
f\ n\ \left\langle \left[param_{n,k}\mapsto\left[C|p\right]_{k}\right],OM,t\right\rangle \ s
\end{gathered}
 & N\left(C\right)+N\left(p\right)=arity_{n}\\
\left\langle \left\langle V,OM,T\right\rangle ,s,\left\langle FP_{n},\left[C|p\right]\right\rangle \right\rangle  & \mbox{otherwise}
\end{cases}\\
\mathtt{bin}\left(binop,c_{0},c_{1}\right) & = & \mbox{perform binary operation between two constants}
\end{eqnarray*}


Note that we largely omit IO operation, since it is not a major concern
of this paper. It is trivial to add another element to the state vector
to take account of IO operations.\caption{Concrete functions}
\label{concfunc}

\end{figure*}



\subsection{Abstract interpretation}

We now define an abstract domain and abstract functions.
\begin{defn}[Abstract domain]
We define an abstract domain of an abstract state, $AState$ as following:

\begin{eqnarray*}
AState & = & \left\langle AVars,AObjMem,AThis,CurriedVals\right\rangle \\
P & = & \left\{ Number,Boolean\right\} \\
Obj & = & \left\{ OBJ_{n}\mid n\in\left\{ 0\right\} \cup\mathbb{N}\right\} \\
FunPointers & = & \left\langle FP_{n},\left[\wp\left(\begin{gathered}P\cup Obj\cup\left\{ FP_{n}\right\} \cup\\
\left\{ \left\langle FC,n,c,eid\right\rangle \right\} 
\end{gathered}
\right)\right]\right\rangle \\
AVal & = & P\cup Obj\cup FunPointers\\
AVars & = & id\rightarrow\wp\left(AVal\right)\\
AObjMem & = & n\rightarrow AVars\\
AThis & = & \wp\left(Obj\right)\\
CurriedVals & = & \left\langle n,c,eid\right\rangle \rightarrow\left[\wp\left(\begin{gathered}P\cup Obj\cup\\
\left\{ \left\langle FC,n,c,eid\right\rangle \right\} 
\end{gathered}
\right)\right]\\
 &  & \mbox{where \emph{n} is a function identifier \emph{c}}\\
 &  & \ \ c\mbox{ is a natural number}
\end{eqnarray*}



In $OBJ_{n}$:


We number each of the ``new'' expressions as we parse the program.
The number n corresponds to the ``new'' expression number.


In $\left\langle FC,n,c,eid\right\rangle $:


n = function number


c = number of arguments curried


eid = A uid of an expression from which this value has been produced

\end{defn}
\begin{example}[Currying loop]
 Consider the following SDTL program.
\end{example}
\begin{lstlisting}
function foo(a,b) {
	return a;
}

x = 0;
while(input < 30) {
	x = foo(x);
}
\end{lstlisting}


In this case, in line 7, x is associated with a value $\left\langle FP_{1},a\mapsto\left\{ Number,\left\langle FC,1,1,e_{1}\right\rangle \right\} \right\rangle $
where $e_{1}$ is the eid of $\mathtt{foo\left(x\right);}$ in line
7. $\nu$ has one entry $\left\langle 1,1,e_{1}\right\rangle \mapsto\left\{ a\mapsto\left\{ 0,\left\langle FC,1,1,e_{1}\right\rangle \right\} \right\} $
\begin{thm}[Finiteness of the abstract property]
Given an SDTL program,$AState$ has a finite domain\end{thm}
\begin{proof}
We show that each of the component $AVars,AObjMem,AThis,CurriedVals$
has a finite domain
\begin{enumerate}
\item First, we show that the number of possible$AVal$ is finite.

\begin{enumerate}
\item Trivially,$P$ is a set of two elements.
\item There are finite number of \emph{new} expressions in a program. Therefore
number of elements in $Obj$ is finite.
\item In case of \emph{FunPointers}

\begin{enumerate}
\item There are a finite number of function declarations. Each function
can only have a set number of parameters. There are a finite number
of expressions in a program. Therefore, number of possible values
of$\left\langle FC,n,c,eid\right\rangle $is finite.
\item Each functions have a set number of accepted parameters. Therefore,
the number of elements in the list is finite.
\item There are only a finite number of function declarations in a program.
\item Hence, \emph{FunPointers} has a finite domain
\end{enumerate}
\item Then, $AVal$, a union of finite domains, also has a finite domain.
\end{enumerate}
\item $AVars$ is a composite of two components: \emph{id} and $\wp\left(AVal\right)$

\begin{enumerate}
\item There are only a finite number of identifiers appearing in a program
\item $AVal$ has a finite domain. Therefore, a powerset of it also has
a finite domain.
\end{enumerate}
\item $AObjMem$ has two components, finiteness of which are argued in 1
(b) and 2 respectively.
\item $AThis$ is a finite as it is a powerset of a finite domain (1 (b))
\item $CurriedVals$ has two components: $\left\langle n,c,eid\right\rangle $
and $\left[\wp\left(P\cup Obj\cup\left\{ \left\langle FC,n,c,eid\right\rangle \right\} \right)\right]$.
Finiteness of both are argued in 1 (c).
\end{enumerate}
\end{proof}
\begin{defn}[Partial ordering]
We define a partial ordering $\left\langle \sigma_{1},\alpha_{1},\tau_{1},\nu_{1}\right\rangle \sqsubseteq\left\langle \sigma_{2},\alpha_{2},\tau_{2},\nu_{2}\right\rangle $
between abstract environments as thus:

\begin{gather*}
\sigma_{1}\sqsubseteq\sigma_{2}\ \mbox{if}\ \forall id\in\mbox{domain}\left(\sigma_{1}\right),\sigma_{1}\left(id\right)\subseteq\sigma_{2}\left(id\right)\\
\alpha_{1}\sqsubseteq\alpha_{2}\ \mbox{if}\ \forall n\in\mbox{domain}\left(\alpha_{1}\right),\alpha_{1}\left(n\right)\sqsubseteq\alpha_{2}\left(n\right)\\
\tau_{1}\sqsubseteq\tau_{2}\ \mbox{if}\ \tau_{1}\subseteq\tau_{2}\\
\nu_{1}\sqsubseteq\nu_{2}\ \mbox{if}\ \forall\left\langle n,c,eid\right\rangle \in\mbox{domain}\left(\nu_{1}\right)\\
\ \ \forall k\in\left\{ 1,2,3,\ldots,c\right\} ,\nu_{1}\left(\left\langle n,c,eid\right\rangle \right)_{k}\subseteq\nu_{2}\left(\left\langle n,c,eid\right\rangle \right)_{k}\\
\left\langle \sigma_{1},\alpha_{1},\tau_{1},\nu_{1}\right\rangle \sqsubseteq\left\langle \sigma_{2},\alpha_{2},\tau_{2},\nu_{2}\right\rangle \equiv\\
\ \ \sigma_{1}\sqsubseteq\sigma_{2}\wedge\alpha_{1}\sqsubseteq\alpha_{2}\wedge\tau_{1}\sqsubseteq\tau_{2}\wedge\nu_{1}\sqsubseteq\nu_{2}
\end{gather*}


\end{defn}
We formulate a least upper bound between two abstract environments.
\begin{thm}
Given two abstract environments $\eta_{1}=\left\langle \sigma_{1},\alpha_{1},\tau_{1},\nu_{1}\right\rangle \ \mbox{and}\ \eta_{2}=\left\langle \sigma_{2},\alpha_{2},\tau_{2},\nu_{2}\right\rangle $,
we can find a least upper bound by finding the value of $\eta_{1}\sqcup\eta_{2}$
\end{thm}
\begin{gather*}
\sigma_{1}\sqcup\sigma_{2}=id\mapsto\sigma_{1}\left(id\right)\cup\sigma_{2}\left(id\right)\\
\alpha_{1}\sqcup\alpha_{2}=n\mapsto\sigma_{1}\left(n\right)\sqcup\sigma_{2}\left(n\right)\\
\tau_{1}\sqcup\tau_{2}=\tau_{1}\cup\tau_{2}\\
\nu_{1}\sqcup\nu_{2}=c\mapsto\nu_{1}\left(c\right)\cup\nu_{2}\left(c\right)\\
\left\langle \sigma_{1},\alpha_{1},\tau_{1},\nu_{1}\right\rangle \sqcup\left\langle \sigma_{2},\alpha_{2},\tau_{2},\nu_{2}\right\rangle \equiv\\
\ \ \left\langle \sigma_{1}\sqcup\sigma_{2},\alpha_{1}\sqcup\alpha_{2},\tau_{1}\sqcup\tau_{2},\nu_{1}\sqcup\nu_{2}\right\rangle 
\end{gather*}

\begin{proof}
Satisfying each lines of conditions gives an environment that can
satisfy conditions set for the partial ordering of both environments.

Observe that if there is an upper bound $\eta'=\left\langle \sigma',\alpha',\tau',\nu'\right\rangle $
of two environments $\eta_{1}$ and $\eta_{2}$, if $\eta'\sqsubseteq\eta_{1}\sqcup\eta_{2}$
then necessarily $\eta'=\eta_{1}\sqcup\eta_{2}$, since any missing
elements in keys and values of $\sigma',\alpha',\tau',\nu'$ will
violate the partial ordering condition.
\end{proof}
We can then deduce that:
\begin{lem}
$\left(AState,\sqsubseteq\right)$ is a chain complete partial ordering.\end{lem}
\begin{defn}[Abstract functions]
 We define abstract functions in figure \ref{absfun}.

\begin{figure*}


\begin{eqnarray*}
\mathtt{isterm}\left(v,s\right) & = & \lambda f,\eta.\begin{cases}
s\ f\ \eta & v=\left\{ Void\right\} \\
\left(s\ f\ \eta\right)\sqcup\left(I_{\rho}\ v\right) & Void\in v,N\left(v\right)>1\\
I_{\rho}\ v & \mbox{otherwise}\left(\mbox{i.e.}\ Void\notin v\right)
\end{cases}\\
\mathtt{cond}\left(v,s_{1},s_{2}\right) & = & \lambda f,\eta.\left(s_{1}\ f\ \eta\right)\sqcup\left(s_{2}\ f\ \eta\right)\\
\mathtt{asg}\left(id,v\right) & = & \lambda\left\langle \sigma,\alpha,\tau,\nu\right\rangle .\left\langle \sigma\left[id=v\right],\alpha,\tau,\nu\right\rangle \\
\mathtt{val}\left(id,v\right) & = & \lambda\left\langle \sigma,\alpha,\tau,\nu\right\rangle .\sigma\left(id\right)\\
\mathtt{get}\left(n,id\right) & = & \lambda\left\langle \sigma,\alpha,\tau,\nu\right\rangle .\left\{ \alpha\left(x\right)\mid OBJ_{x}\in n\right\} \\
\mathtt{set}\left(n,id,v\right) & = & \lambda\left\langle \sigma,\alpha,\tau,\nu\right\rangle .\left\langle \sigma,\left\{ x\mid{OBJ_{x}\in\alpha,x=\begin{cases}
\alpha\left(x\right)\left[id\mapsto v\right] & OBJ_{x}\in n\\
\alpha\left(x\right) & \mbox{otherwise}
\end{cases}}\right\} ,\tau,\nu\right\rangle \\
\mathtt{conval}\left(con\right) & = & \begin{cases}
Number & con\in\mathbb{N}\\
Boolean & con\in\left\{ true,false\right\} 
\end{cases}\\
\mathtt{obj}\left(j\right) & = & \lambda\left\langle \sigma,\alpha,\tau,\nu\right\rangle .\alpha\left(j\right)\\
\mathtt{getthis} & = & \lambda\left\langle \sigma,\alpha,\tau,\nu\right\rangle .\tau\\
\mathtt{getinput} & = & \lambda\eta.\left\langle \eta,\left\{ Number\right\} \right\rangle \\
\mathtt{dooutput}\left(v\right) & = & \lambda\eta.\eta\\
\mathtt{newobj} & = & \lambda\left\langle \sigma,\alpha,\tau,\nu\right\rangle .\left\langle \left\langle \sigma,\alpha\left[n\mapsto\phi\right],\tau,\nu\right\rangle ,OBJ_{n},\nu\right\rangle \ \mbox{where}\ n=N\left(\alpha\right)\\
\mathtt{call}\left(n,p,t,e\right) & = & \lambda f,s,\left\langle \sigma,\alpha,\tau,\nu\right\rangle .\underset{\left\langle FP_{x},C\right\rangle \in n}{\bigsqcup}\mathtt{scall}\left(\left\langle FP_{x},C\right\rangle ,p,t,f,\left\langle \sigma,\alpha,\tau,\nu\right\rangle ,s,e\right)\\
{\scriptstyle \mathtt{scall}\left(\left\langle FP_{n},C\right\rangle ,p,t,f,\left\langle \sigma,\alpha,\tau,\nu\right\rangle ,s,e\right)} & = & \begin{cases}
\begin{gathered}\left\langle \left\langle \sigma,\alpha',\tau,\nu'\right\rangle ,s',r'\right\rangle \ \mbox{where}\ \left\langle \left\langle \sigma',\alpha',\tau'\right\rangle ,s',r,\nu'\right\rangle =\\
f\ n\ \left\langle \left\{ param_{n,k}\mapsto\left[C|p\right]_{k}\right\} ,\alpha,\tau,\nu\right\rangle \ s
\end{gathered}
 & N\left(C\right)+N\left(p\right)=arity_{n}\\
\left\langle \left\langle \sigma,\alpha,\tau,\nu\left[\left\langle e,n,N\left(C\right)+N\left(p\right)\right\rangle \mapsto\left[C|p\right]\right]\right\rangle ,s,\left\langle FP_{n},\left[C|p\right],e\right\rangle \right\rangle  & \mbox{otherwise}
\end{cases}\\
\mathtt{bin}\left(binop,c_{0},c_{1}\right) & = & \begin{cases}
\left\{ Number\right\}  & binop\in\left\{ '+','-','*','/'\right\} \\
\left\{ Boolean\right\}  & \mbox{otherwise}
\end{cases}
\end{eqnarray*}


\caption{Abstract functions}
\label{absfun}
\end{figure*}

\end{defn}
In order to argue for a completeness of an analysis, we first argue
monotonicity of the interpretation.
\begin{thm}
\label{thm:mono}Abstract functions produce monotonic lambda functions
\begin{proof}
Following list of arguments cover all the functions:
\begin{enumerate}
\item Observe that for the functions calculating values from given parameters
and an environment, an increase in the environment implies that the
values we obtain are also increased. This is from the definition of
the partial ordering.
\item For the functions that are updating environment, observe that an increase
in the environment implies an increase in the updated environment.
\item In case of $\mathtt{call}$, its the least upper bound of results
from $\mathtt{scall}$, which falls into categories 1 and 2. An increase
in the results implies increase in the least upper bound of them.
\end{enumerate}
\end{proof}
\end{thm}
From theorems \ref{thm:bindmono} and \ref{thm:mono} give the following
important lemma:
\begin{lem}
Abstract $\mathcal{S},\mathcal{E}\ \mbox{and}\ \mathcal{L}$ are monotonic
functions.
\end{lem}
Then, it stands to reason that fixed points exist for our analytic
framework.
\begin{lem}
Fixed points exist for an abstract $\mathcal{F}$
\end{lem}
Hence, it stands to reason that for any given program, we obtain an
analysis of it.

Now we turn to examine the relationship between concrete and abstract
domain.
\begin{defn}[Abstraction]
 We define an abstraction relation between concrete and abstract
environments, $\left\langle \sigma,\alpha,\tau,\nu\right\rangle \succ\left\langle V,OM,T\right\rangle $
as following:

\begin{gather*}
\left\langle \sigma,\alpha,\tau,\nu\right\rangle \succ\left\langle V,OM,T\right\rangle \ \mbox{iff}\ \\
\ \ \forall id\in V\exists a\in\sigma\left(id\right),a\succ V\left(id\right)\mbox{ and }\tau\succ T\\
\forall z\in\mathbb{Z},\left\{ Number\right\} \succ z\\
\forall b\in\left\{ True,False\right\} ,\left\{ Boolean\right\} \succ b\\
OBJ_{m}\succ OBJ_{n}\ \mbox{if}\ \eta\vdash OBJ_{m}\succ OM\vdash OBJ_{n}\\
\left\langle FP_{m},\left[c'\right],e\right\rangle \succ\left\langle FP_{n},\left[c\right]\right\rangle \ \mbox{if}\ m=n\ \\
\ \ N\left(c\right)=N\left(c'\right)\ \mbox{and}\ \forall c_{k}\in c,\eta\vdash c'_{k}\succ c_{k}
\end{gather*}


In turn, contextual definitions of abstract object and function pointer
values are defined as following:

\begin{gather*}
\left\langle \sigma,\alpha,\tau,\nu\right\rangle \vdash OBJ_{m}\succ OBJ_{n}\mbox{ if }\\
\forall id\in OM\left(n\right),OM\left(n\right)\left(id\right)\in Z\cup\left\{ True,False\right\} ,\\
\ \ \alpha\left(m\right)\left(id\right)\succ OM\left(n\right)\left(id\right)\\
\forall id\in OM\left(n\right),OM\left(n\right)\left(id\right)=OBJ_{x},\alpha\left(m\right)=OBJ_{y},\\
\ \ \alpha\left(y\right)\succ OBJ_{x}
\end{gather*}

\end{defn}
We now define abstraction and concretisation relations between an
abstract domain and a powerset of concrete domain.
\begin{defn}[Abstraction and concretisation]
 We define abstraction function $P$ and concretisation function
$Q$ as following:

\begin{gather*}
P:\wp\left(CState\right)\rightarrow AState\\
P\left(r\right)=\bigsqcup\left\{ \eta|\rho\in r,\eta\succ\rho\right\} \\
Q:AState\rightarrow\wp\left(CState\right)\\
Q\left(\eta\right)=\left\{ r|r\in\rho,\eta\succ\rho\right\} 
\end{gather*}


Note that, as one would expect, we find a Galois connection between
the abstract domain and the collection of concrete states.\end{defn}
\begin{cor}[Galois connection]
\label{thm:galois}$\left(P,Q\right)$ forms a Galois connection
between $\left(AState,\sqsubseteq\right)$ and$\left(\wp\left(CState\right),\subseteq\right)$
\end{cor}
Now we turn to argue isomorphism. For the simplicity of argument,
without losing generality, we lift abstract functions to produce a
lambda function of the type $\left(State\times StateCol\right)\rightarrow\left(State\times StateCol\times a\right)$.
\begin{thm}[Isomorphism]
For all abstract and concrete functions, the following holds true:
when $F\left(n_{c},id_{c},v_{c},p_{c},t_{c},e_{c},j_{c}\right)=f,f\left(s_{a},\rho\right)=\left\langle s_{c}',\rho',V_{C}\right\rangle ,G\left(n_{a},id_{a},v_{a},p_{a},t_{a},e_{a},j_{a}\right)=g,g\left(s_{a},\eta\right)=\left\langle s_{a}',\eta',V_{A}\right\rangle $ 

\begin{multline*}
\left(\begin{array}{ccc}
\eta\vdash n_{a} & \succ & \rho\vdash n_{c}\\
id_{a} & = & id_{c}\\
\eta\vdash v_{a} & \succ & \rho\vdash v_{c}\\
\eta\vdash p_{a,k} & \succ & \rho\vdash p_{c,k}\mbox{ for all\ }k\\
\eta\vdash t_{a} & \succ & \rho\vdash t_{c}\\
e_{a} & = & \rho\vdash e_{c}\\
\eta\vdash OBJ_{j_{a}} & \succ & \rho\vdash OBJ_{j_{c}}
\end{array}\right)\implies\\
\left(\left\langle s_{a},\eta\right\rangle \succ\left\langle s_{c},\rho\right\rangle \implies\left\langle s_{a}',\eta',V_{A}\right\rangle \succ\left\langle s_{c}'\rho',V_{C}\right\rangle \right)
\end{multline*}


(ignoring conditions set for parameters that the function is not taking)\end{thm}
\begin{proof}
As with \ref{thm:mono}, we categorise our discussion as following:
\begin{enumerate}
\item Observe that for the functions calculating values from given parameters
and an environment, isomorphicity between parameters and given environments
implies isomorphicity of the results.
\item For the functions that are updating environment, observe that an isomorphicity
between given environments and between parameters imply an isomorphicity
in the updated environments.
\item In case of $\mathtt{call}$, its the least upper bound of results
from $\mathtt{scall}$, which falls into categories 1 and 2. Given
Theorem \ref{thm:galois}, when $\mathtt{scall}$ functions are isomorphic,
then so are $\mathtt{call}$ functions.
\end{enumerate}
\end{proof}
These theorems together with the properties of bind operator give
rise to an important lemma in proving the correctness of the analysis.
That is:
\begin{lem}
Abstract $\mathcal{F},\mathcal{S},\mathcal{E},\mathcal{L}$ and concrete
$\mathcal{F},\mathcal{S},\mathcal{E},\mathcal{L}$ are isomorphic.
\end{lem}
Now we are ready to argue correctness.
\begin{thm}[Correctness of the analysis]
 Given an SDTL program, 

If $\mathcal{F}\ 0\ \left\langle \phi,\left\{ 0\mapsto\phi\right\} ,OBJ_{0}\right\rangle \ \phi=\rho'\ S_{C}$
and $\mathcal{F}\ 0\ \left\langle \phi,\left\{ 0\mapsto\phi\right\} ,\left\{ OBJ_{0}\right\} ,\phi\right\rangle \ \phi=\eta'\ S_{A}$
then $S_{C}$ and $S_{A}$ are related in the following manner:

$\forall sid\in P,\exists\rho_{0},\rho_{1},\rho_{0}\times\rho_{1}\in S_{A}\left(sid\right)\rightarrow\exists\eta_{0},\eta_{1},\eta_{0}\times\eta_{1}\in S_{A}\left(sid\right)\wedge\eta_{0}\succ\rho_{0}\wedge\eta_{1}\succ\rho_{1}$
where $P$ is set of all statement IDs in the program.\end{thm}
\begin{proof}
Recall the isomorphicity between abstract and concrete functions.
If there is a state collected in the concrete result, then it is implied
that an abstraction of such state is collected in an abstract result.
\end{proof}

\subparagraph{Dual of the analysis}

Partial ordering theory provides that what holds true for the particular
partial ordering will also hold true for a dual of it. We show the
dual of this analysis by the following example.
\begin{example}[Dual]
Consider the following program.

\begin{lstlisting}
function object() { this.value = 1; }

function foo(x) {
	if(input > 20) { 
		x.propA = 20;
		x.propC = 15;
	else {
		x.propB = False;
		x.propC = 15;
	}
}

z = new object(); # 1

if(input > 15) {
	a = new object(); # 2
	foo(a);
} else {
	a = new object(); # 3
	foo(a);
}

output a.value + a.propC;
\end{lstlisting}


we obtain different results depending on the kind of analysis we perform.\end{example}
\begin{itemize}
\item $\subseteq$ ordering
\end{itemize}
\begin{eqnarray*}
\sigma_{\subseteq} & = & \left\{ \begin{alignedat}{2}z & \mapsto & \left\{ OBJ_{1}\right\} \\
a & \mapsto & \left\{ OBJ_{2},OBJ_{3}\right\} 
\end{alignedat}
\right\} \\
\alpha_{\subseteq} & = & \left\{ \begin{alignedat}{2}1 & \mapsto & \left\{ \mathtt{value}\mapsto\left\{ Number\right\} \right\} \\
2,3 & \mapsto & \left\{ \begin{gathered}\mathtt{value}\mapsto\left\{ Number\right\} \\
\mathtt{propA}\mapsto\left\{ Number\right\} \\
\mathtt{propB}\mapsto\left\{ Boolean\right\} \\
\mathtt{propC}\mapsto\left\{ Number\right\} 
\end{gathered}
\right\} 
\end{alignedat}
\right\} 
\end{eqnarray*}

\begin{itemize}
\item $\supseteq$ ordering
\end{itemize}
\begin{eqnarray*}
\sigma_{\supseteq} & = & \left\{ \begin{alignedat}{2}z & \mapsto & \left\{ OBJ_{1}\right\} \\
a & \mapsto & \left\{ \right\} 
\end{alignedat}
\right\} \\
\alpha_{\supseteq} & = & \left\{ \begin{alignedat}{2}1 & \mapsto & \left\{ \mathtt{value}\mapsto\left\{ Number\right\} \right\} \\
2,3 & \mapsto & \left\{ \begin{gathered}\mathtt{value}\mapsto\left\{ Number\right\} \\
\mathtt{propC}\mapsto\left\{ Number\right\} 
\end{gathered}
\right\} 
\end{alignedat}
\right\} 
\end{eqnarray*}


Given our treatment of each new expression as a creation of a type
of its own, the result that we obtain for $\sigma_{\supseteq}$ might
be too limiting and imprecise. We could use the combination of $\sigma_{\subseteq}$
and $\alpha_{\supseteq}$ to calculate a member of an object that
is referred to by a variable must have. Combinations of results from
may and must analysis is helpful in designing an IDE for a programming
language, where the IDE can statically inform the programmer what
members would an object must have or may have at a particular program
point.


\section{Conclusion and Future Direction}

As we have noted in the introduction, this work is a snapshot of an
ongoing dialect between theory and practice, positively informing
each other to gradually move towards a better theorisation (and practical
implementation) of a difficult task of analysing dynamic languages.
We sought to modularise the theoretic framework so that we can take
an evolutionary approach. We anticipate that, as a result of having
such theory, adding new features to the current SDTL while maintaining
formality and rigour will be considerably less laborious than to invent
a new incarnation of a more feature-complete model language and produce
theory for it.

\bibliographystyle{plain}
\bibliography{duck}

\end{document}
