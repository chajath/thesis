\NeedsTeXFormat{LaTeX2e}
\documentclass{CRPITStyle}

\usepackage{aopmath}
\usepackage{amssymb}
\usepackage{harvard}
\usepackage{array}
\usepackage{pstricks}
\usepackage{pst-node}
\usepackage{graphicx}

% should use better math package...
\newtheorem{definition}{Definition}
\newenvironment{proof}
  {\begin{trivlist}\item[]\textbf{Proof}}
  {\hspace*{1em}\halmos \end{trivlist}}
% \newtheorem{proposition}{Proposition}
% \newtheorem{observation}{Observation}
% end-of-proof from JLP
\def\halmos{\mbox{\hspace*{1pc}$\Box$}}

\renewcommand{\mathindent}{5pt}
\renewcommand{\arraystretch}{1.1}
\setlength{\extrarowheight}{2pt}

\newcommand{\authornote}[3]{
% Uncomment next and third line to make author notes conditional on draft mode
%\ifthenelse{\lengthtest{\overfullrule = 5pt}}%
    {\fbox{\sc #1}:\textcolor{#2}{\textbf{#3}}}%
%    {}%
}

\newcommand{\lee}[1]{\authornote{pjs}{blue}{#1}}
\newcommand{\hcs}[1]{\authornote{hcs}{red}{#1}}

\newcommand{\ignore}[1]{}

\newcommand{\Bi}{\ensuremath{\mathbf{2}}}
\newcommand{\Tri}{\ensuremath{\mathbf{3}}}
\newcommand{\Quad}{\ensuremath{\mathbf{4}}}
\newcommand{\ident}[1]{\mbox{\textit{#1}}}
\newcommand{\pset}[1]{{\cal P}(#1)}
\newcommand{\fun}{\rightarrow}
\newcommand{\invis}[1]{\phantom{#1}}
\newcommand{\derives}{\stackrel{*}{\Rightarrow}}
\newcommand{\false}{\ensuremath{\mathbf{f}}\xspace}
\newcommand{\true}{\ensuremath{\mathbf{t}}\xspace}
\newcommand{\impl}{\mathbin{\Rightarrow}}
\newcommand{\biim}{\mathbin{\Leftrightarrow}}
\newcommand{\nat}{\mathbb{N}}
\newcommand{\intg}{\mathbb{Z}}
\newcommand{\rational}{\mathbb{Q}}
\newcommand{\real}{\mathbb{R}}
\newcommand{\Id}{\mathbf{Id}}
\newcommand{\Val}{\mathbf{Val}}
\newcommand{\Pred}{\mathbf{Pred}}
\newcommand{\Env}{\mathbf{Env}}
\newcommand{\Her}{\mathcal{G}}
\newcommand{\Interp}{\mathcal{I}}
\newcommand{\lfp}{\mathbf{lfp}}
\newcommand{\gfp}{\mathbf{gfp}}

\newcommand{\sset}[2]{\left\{~#1 \left|
      \begin{array}{l}#2\end{array}
    \right.     \right\}}

\def\lbag       {\lbrack\!\lbrack}
\def\rbag       {\rbrack\!\rbrack}
\def\lquote     {[\![}
\def\rquote     {]\!]}
\def\dlangle    {\langle\!\langle}
\def\drangle    {\rangle\!\rangle}
\newcommand{\qq}[1]{\lquote #1 \rquote}
\newcommand{\tuple}[1]{\ensuremath{\langle #1 \rangle}}

% CATS specific details:
\newcommand\conferencenameandplace{19th Computing: Australasian Theory Symposium (CATS 2013), Adelaide, Australia, January 2013}
\newcommand\volumenumber{xxx}
\newcommand\conferenceyear{2013}
\newcommand\editorname{Anthony Wirth}

\begin{document}

\title{Type Analysis for a Duck-Typed Language with Reflection}

\author{In-Ho Yi, Peter Schachte, and Harald S{\o}ndergaard\\
  Department of Computing and Information Systems \\
  The University of Melbourne, Victoria 3010,
  Australia \\
  {i.yi@student.unimelb.edu.au,
    schachte@unimelb.edu.au,
    harald@unimelb.edu.au
  }
}

\maketitle
\toappear{Copyright \copyright 2013, Australian Computer Society, Inc.
This paper appeared at the
\emph{19th Computing: The Australasian Theory Symposium} (CATS 2013),
Adelaide, Australia.
Conferences in Research and Practice in Information Technology, Vol. xxx.
Anthony Wirth, Ed.
Reproduction for academic, not-for profit purposes permitted provided
this text is included.}

\maketitle

% \toappearstandard
% \toappearNONstandard

\begin{abstract}

We construct an analytic framework for static analysis of a simple
duck-typed language, and we exemplify its use by developing an
analysis to 

\end{abstract}

\section{Introduction}

Recent years have seen the growing interest on dynamic typing languages.
Traditionally these languages were termed 􏰁scripting languages,
as they were mainly used for automating tasks and processing strings.
However, with the advent of Web environment, languages such as Perl and 
PHP gained popularity as languages for web application development. 
On a client side, web pages make heavy use of JavaScript, 
a dynamic typing language, to deliver dynamic contents to the browser. 
Recent years have seen increasing use of JavaScript in server side 
environment, as well.

What these languages provide is an ability to rapidly prototype and 
validate application models in a real time REPL environment. 
Another strength comes from the fact that programmers do not need 
to have a class structure defined up front.
Rather, class structures and types of variables in general are 
dynamically built. 
This reduces an initial overhead of software design.

However, these features come at a cost. 
Lacking formal and static de􏰄nition of type information makes 
dynamically typing languages harder to analyse. 
This di􏰅culty causes several practical problems.

First, it has been noted that as applications become more mature, 
more efforts are devoted to program unit testing or assertions 
to make sure type safety of systems. 
These extra overheads can sometimes overshoot the benefit of having
a dynamically typing language.

Second, whereas programmers writing statically typed languages enjoy 
the abundance of development tools, tools aiding development of 
dynamically typed languages are lacking powers, 
largely due to the difficulty and sometimes infeasibility of conducting
a type analysis.

Finally, not having a static type structure has a significant performance
implication. 
There are ongoing development in the industry at large to overcome and 
improve the performance of dynamically typing languages.

With these problems in mind, the aim of this paper is to develop an 
analytic framework for a type analysis of a dynamic typing language. 
We design a toy language that showcases ``dynamism'' built in some 
of the aforementioned scripting languages, such as duck-typing, 
reflection and function currying. 
A notable omission is a closure scoping. 
However, allowing function currying gives enough expressive power 
to the toy language.

We then develop a framework that, depending on definitions of 
abstract functions, can function as both denotational semantics 
and abstract interpretation. 
The focus of this effort is to build a theoretical framework where 
much of the proof of correctness is built in to the system, 
thereby allowing theoreticians to avoid reproducing large body of 
work in both semantics and abstract interpretation, and focus at 
the core of the problem, that is the definition of abstract domains
and their correlation with the concrete environment.

\section{Related Work}

These seem relevant: 
\citeasnoun{Bono:arXiv2012},
\citeasnoun{Bloom:OOPSLA2009},
\citeasnoun{Wrigstad:POPL2010}.

Liang and Hudak
also give ``monadic'' denotational definitions of programming 
languages~\cite{Liang_Hudak:ESOP1996,Liang:phd1998}.
However, their aim is to show that the approach makes it easier
to extend a denotational definition to account for new features
in the language being defined.
A well-known problem with denotational semantics is
the fact that incorporation of new object language features may require
extensive rewriting of \emph{existing} definitions,
even for simple extensions.
For example, a semantic function 
${\cal E}: \ident{Exp} \rightarrow \ident{Val}$ 
may have been defined for a language of arithmetic expressions.
If we want to extend the language to allow for the use of variables
in expressions, the type of ${\cal E}$ needs to change to, say,
$\ident{Exp} \rightarrow 
(\ident{Var} \rightarrow \ident{Val}) \rightarrow \ident{Val}$,
and all the equations defining ${\cal E}$ similarly needs to be
changed to pass around a ``store'' of type 
$\ident{Var} \rightarrow \ident{Val}$.
Liang and Hudak show how a monadic approach can 



\bibliographystyle{agsm}
\bibliography{duck}
\end{document}
